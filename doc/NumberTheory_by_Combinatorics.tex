\documentclass{article}
\usepackage{amssymb,amsmath,amsthm,url}
\usepackage{fullpage}
\theoremstyle{definition}
\newtheorem{proposition}{Proposition}
\newtheorem{theorem}{Theorem}
\newtheorem{remark}{Remark}
%\def\proofname{証明}
\title{Combinatorial Proofs for Some Number-Theoretic Facts}
\author{Koji Nuida}
\date{June 7, 2023 (first version)}
\begin{document}
\maketitle
\vspace*{-2em}

\begin{abstract}
    \lq\lq Combinatorial proof\rq\rq{} means a proof of equation for non-negative integers by counting the number of elements in some finite set in two different ways.
    In this note, we describe combinatorial proofs for some facts in number theory.
\end{abstract}


\section*{Notations}

Let $\mathbb{Z}_{>0}$ denote the set of positive integers, and let $\mathbb{Z}_{\geq 0}$ denote the set of non-negative integers.
For $n,m \in \mathbb{Z}$, we define $[n,m] := \{ k \in \mathbb{Z} \mid n \leq k \leq m \}$.
For a set $S$ and $n \in \mathbb{Z}_{\geq 0}$, we write the set of all $n$-element subsets of $S$ as $\binom{S}{n}$.
That is, $\binom{S}{n} = \{ T \subseteq S : |T| = n \}$.
For $n \in \mathbb{Z}_{>0}$ and $a,b \in \mathbb{Z}$, we write $a \equiv_n b$ to mean $a \equiv b \pmod{n}$.
Moreover, let $a \bmod n$ denote the remainder of $a \in \mathbb{Z}$ modulo $n \in \mathbb{Z}_{>0}$.


\section{Warm-Up: Expression of Binomial Coefficients}

First, as an example of the methodology of combinatorial proofs itself, we describe a proof for the explicit expression of binomial coefficients.
Here, for non-negative integers $n,m \in \mathbb{Z}_{\geq 0}$, we define the binomial coefficient $\binom{n}{m}$ to be the number of the $m$-element subsets of an $n$-element set (e.g., $[1,n]$).
By using the notation above, it can be expressed as $\binom{n}{m} = | \binom{[1,n]}{m} |$.
This value is, by definition, a non-negative integer.
We describe a combinatorial proof of the following well-known expression of binomial coefficients.
We note that if $m > n$, then $\binom{n}{m} = 0$.

\begin{proposition}
    \label{prop:binomial_coefficient_def}
    If $n,m \in \mathbb{Z}_{\geq 0}$ and $m \leq n$, then $\displaystyle \binom{n}{m} = \frac{ n! }{ m! (n-m)! }$.
\end{proposition}
\begin{proof}
    We enumerate the elements of the $n$-th symmetric group $S_n$ in two ways.
    First, for $\sigma \in S_n$, there are $n$ choices for $\sigma(1)$, there are $n-1$ choices for $\sigma(2)$, there are $n-2$ choices for $\sigma(3)$, and so on, and hence we have $|S_n| = n!$.

    Secondly, we consider the following way of enumeration: (i) choose the set $I$ of $m$ numbers $\sigma(1),\dots,\sigma(m)$; (ii) determine the order of elements of $I$; and (iii) determine the order of the remaining elements not in $I$.
    There are $\binom{n}{m}$ choices for (i) by the definition of binomial coefficients.
    For each of them, there are $m!$ choices for (ii) and $(n-m)!$ choices for (iii).
    As these numbers are independent of $I$, the total number of elements of $S_n$ is equal to $\binom{n}{m} \cdot m! (n-m)!$.

    As a result, we have $n! = |S_n| = \binom{n}{m} \cdot m! (n-m)!$, which implies the claim by dividing both sides by $m! (n-m)!$.
\end{proof}


\section{Fermat's Little Theorem}
\label{sec:Fermat's_Little_Theorem}

The statement of Fermat's Little Theorem is as follows (which is one of the equivalent formulations).

\begin{theorem}
    [Fermat's Little Theorem]
    \label{thm:Fermat's_Little_Theorem}
    Let $p$ be a prime and $a \in \mathbb{Z}$.
    Then $a^p \equiv_p a$.
\end{theorem}

This is a famous result in elementary number theory, and some well-known proofs are one using the multiplicative group of the finite field $\mathbb{F}_p$ and one by mathematical induction using the expansion of $(a+1)^p$.
Here we describe a combinatorial proof.

\begin{proof}
    We may assume without loss of generality that $a > 0$, by adding some multiple of $p$ to $a$ if necessary.
    It suffices to show that $a^p - a$ is a multiple of $p$.

    Let $X := [1,a]^p$, i.e., the set of sequences of length $p$ on the set $\{ 1, 2, \dots, a\}$.
    We have $|X| = a^p$.

    On the other hand, we consider the cyclic shift operation $\sigma$ on sequences $x = (x_1,x_2,\dots,x_{p-1},x_p) \in X$ defined by $\sigma(x) = (x_2,x_3,\dots,x_p,x_1) \in X$.
    This is a permutation on $X$ with $\sigma^p = \mathsf{id}$.
    To analize the orbit decomposition of $X$ by the action of the group $G := \langle \sigma \rangle$ of order $p$, we say that $x \in X$ is of type 1 if $\sigma(x) = x$, and of type 2 if $\sigma^k(x) \neq x$ for any $k \in [1,p-1]$ (note that both cannot be simultaneously satisfied, as $p \geq 2$).
    Now assume that there is an $x \in X$ not of type 1 nor type 2.
    As $x$ is not of type 2, there is a $k \in [1,p-1]$ with $\sigma^k(x) = x$; we choose such a minimum $k$.
    As $x$ is not of type 1 either, we have $2 \leq k \leq p-1$.
    As $p$ is prime, $p$ is not a multiple of $k$, and by dividing $p$ by $k$, we have $p = qk + r$ for some $q \in \mathbb{Z}_{\geq 0}$ and $r \in [1,k-1]$.
    Now $\sigma^k(x) = x$ and hence $\sigma^{qk}(x) = x$ by the choice of $k$, while $\sigma^p(x) = \sigma^{qk+r}(x) = x$.
    Comparing them implies that $\sigma^r(x) = x$, contradicting the minimality of $k$, as $1 \leq r < k$.
    Hence, each element of $X$ is either of type 1 or of type 2.

    For $x \in X$, being of type 1 is equivalent to that all components are equal, therefore the number of such elements of $X$ is $a$.
    Hence the number of elements in $X$ of type 2 is $a^p - a$.
    On the other hand, the set $X_2$ of elements in $X$ of type 2 is invariant under the action of $G$, and each $x \in X_2$ has trivial fixing subgroup $G_x := \{ \tau \in G \mid \tau(x) = x \} = \{\mathsf{id}\}$.
    Therefore $X_2$ is decomposed into the $G$-orbits each having cardinality $|G| = p$, implying that $|X_2| \equiv_p 0$.
    Hence we have $a^p - a \equiv_p 0$, as desired.
\end{proof}


\section{On Divisors of Binomial Coefficients}

\begin{proposition}
    \label{prop:binomial_divisible_by_n}
    For $n,m \in \mathbb{Z}_{>0}$, if $n$ is coprime to $m$, then $\binom{n}{m}$ is a multiple of $n$.
\end{proposition}

A special case of this proposition is a well-known fact that if $p$ is prime and $k \in [1,p-1]$, then $\binom{p}{k}$ is a multiple of $p$.
We note that the proof for Fermat's Little Theorem by mathematical induction using the expansion of $(a+1)^p$, briefly mentioned in Section \ref{sec:Fermat's_Little_Theorem}, uses this fact, while our combinatorial proof above did not require this fact.

\begin{proof}
    Let $X := \binom{[1,n]}{m}$.
    Then $|X| = \binom{n}{m}$ by the definition of binomial coefficients.

    Let $\sigma$ denote the cyclic permutation $(1\;2\;\cdots\;n) \in S_n$ of length $n$.
    Then $G := \langle \sigma \rangle$ acts on $X$ by $\sigma \cdot S = \{ \sigma(s_1),\dots,\sigma(s_m) \}$ for $S = \{s_1,\dots,s_m\} \in X$.
    Each orbit of $X$ by this action has order at most $|G| = n$.
    If each orbit has order precisely $n$, then the orbit decomposition implies that $|X| = \binom{n}{m}$ is a multiple of $n$, as desired.
    From now, we assume that there is an orbit in $X$ with order less than $n$ and deduce a contradiction.
    Let $S \in X$ be an element of this orbit.

    By the choice of $S$, there is a $k \in [1,n-1]$ with $\sigma^k \cdot S = S$.
    We choose such a minimum $k$.
    Then $\sigma^k(a) \in S$ for each $a \in S$.
    Now by dividing $n$ by $k$, we have $n = qk + r$ for some $q \in \mathbb{Z}_{\geq 0}$ and $r \in [0,k-1]$.
    For each $a \in S$, we have $\sigma^n(a) = a$ by the definition of $\sigma$, therefore $\sigma^{n+k-r}(a) = \sigma^{k-r}(a)$; while we have $n + k - r = (q+1)k$ and therefore $\sigma^k \cdot S = S$ by the choice of $k$, implying that $\sigma^{n+k-r}(a) \in S$.
    Hence we have $\sigma^{k-r}(a) \in S$.
    This implies that $\sigma^{k-r} \cdot S = S$, which contradicts the minimality of $k$ if $r > 0$.
    Therefore we have $r = 0$ and $k$ is a divisor of $n$.
    Let $\tau := \sigma^k$ and $d := n/k$.
    Then $\tau^d = \sigma^n = \mathsf{id}$.
    Moreover, for any $a \in S$ and $\ell \in [1,d-1]$, as $1 \leq k \cdot \ell < n$, we have $\tau^{\ell}(a) = \sigma^{k \cdot \ell}(a) \neq a$ by the definition of $\sigma$.
    This implies that each orbit of $S$ by the action of $H := \langle \tau \rangle$ consists of precisely $d$ elements, therefore $|S|$ is a multiple of $d$.
    However, now $|S| = m$ is coprime to $n$ and $d$ is a divisor of $n$ with $d > 1$, a contradiction.
    This concludes the proof.
\end{proof}

We note that the converse of Proposition \ref{prop:binomial_divisible_by_n} does not hold; $\binom{10}{4} = \frac{ 10 \cdot 9 \cdot 8 \cdot 7 }{ 4 \cdot 3 \cdot 2 \cdot 1 } = 210$ gives a counterexample.


\section{Lucas' Theorem}

Lucas' Theorem \cite{Lucas78} in elementary number theory is stated as follows.
Here we describe a combinatorial proof.

\begin{theorem}
    [Lucas' Theorem]
    \label{thm:Lucas'_Theorem}
    Let $p$ be a prime and let $d \in \mathbb{Z}_{>0}$.
    Suppose that $n,m \in \mathbb{Z}_{\geq 0}$ can be expressed by $d$-digit $p$-ary expressions, say $n = (n_{d-1} n_{d-2} \cdots n_0)_p$, $m = (m_{d-1} m_{d-2} \cdots m_0)_p$ (where $n_i, m_i \in [0,p-1]$ and the most significant digits may be $0$).
    Then we have
    \[
        \binom{n}{m} \equiv_p \binom{n_{d-1}}{m_{d-1}} \binom{n_{d-2}}{m_{d-2}} \cdots \binom{n_0}{m_0} \enspace.
    \]
\end{theorem}
\begin{proof}
    Let $X := \binom{[0,n-1]}{m}$.
    We have $|X| = \binom{n}{m}$ by the definition of binomial coefficients.

    For $\ell \in [0,d-1]$ and $\alpha \in [0,n_{\ell}-1]$, we define
    \[
        Y(\ell, \alpha) := \{ (n_{d-1} \cdots n_{\ell+1} \alpha *_{\ell-1} \cdots *_1 *_0)_p \in \mathbb{Z}_{\geq 0} \mid \mbox{$*_i \in [0,p-1]$ for any $i \in [0,\ell-1]$} \} \enspace.
    \]
    They are disjoint and form a partition of $[0,n-1]$.
    For $k \in [0,d-2]$ and $x \in \mathbb{Z}_{\geq 0}$, we define $f_k(x)$ to be the number obtained by changing the $k$-th or lower digits $x_k$, $\dots$, $x_1$, $x_0$ to $p-1$ in the $p$-ary expression $x = (\cdots x_2 x_1 x_0)_p$.
    Moreover, for $x \in Y(\ell,\alpha)$, we define $\sigma_k(x)$ in a way that if $f_k(x) \leq n-1$, then $\sigma_k(x)$ is obtained by changing the $k$-th digit $x_k$ of $x$ to $x_k + 1 \bmod p$, and if $f_k(x) > n-1$, then $\sigma_k(x) = x$.
    Now for any $x \in Y(\ell,\alpha)$, if $k \leq \ell-1$, then we have $f_k(x) \leq f_{\ell-1}(x) = (n_{d-1} \cdots n_{\ell+1} (\alpha+1) 0 \cdots 00)_p - 1 \leq n-1$, therefore $x$ is not fixed by $\sigma_k$, and $\sigma_k(x) \in Y(\ell,\alpha)$ by the definition of $Y(\ell,\alpha)$.
    On the other hand, if $k \geq \ell$, then we have $f_k(x) \geq f_{\ell}(x) = (n_{d-1} \cdots n_{\ell+1} (p-1) \cdots (p-1)(p-1))_0 \geq n > n-1$, therefore $\sigma_k(x) = x$.
    This implies that the set $Y(\ell,\alpha)$ is invariant under any $\sigma_k$; each of $\sigma_{\ell},\dots,\sigma_{d-2}$ fixes every element of $Y(\ell,\alpha)$, while each of $\sigma_0,\dots,\sigma_{\ell-1}$ fixes no element of $Y(\ell,\alpha)$.
    By this and the fact that $[0,n-1]$ is partitioned into the subsets $Y(\ell,\alpha)$, it follows that each $\sigma_k$ is a permutation on $[0,n-1]$ with $\sigma_k^p = \mathsf{id}$.

    We show that if $\ell_1 < \ell_2$, then $\sigma_{\ell_1}$ and $\sigma_{\ell_2}$ commute with each other.
    Indeed, for $x \in Y(\ell,\alpha)$, the argument in the previous paragraph implies the following:
    \begin{itemize}
        \item When $\ell > \ell_2$, we also have $\ell > \ell_1$.  Therefore, both $(\sigma_{\ell_1} \circ \sigma_{\ell_2})(x)$ and $(\sigma_{\ell_2} \circ \sigma_{\ell_1})(x)$ are obtained by adding $1$ (modulo $p$) to each of the $\ell_1$-th and the $\ell_2$-th digits of $x$, where they differ only in the order of the two additions.  Hence we have $(\sigma_{\ell_1} \circ \sigma_{\ell_2})(x) = (\sigma_{\ell_2} \circ \sigma_{\ell_1})(x)$.
        \item When $\ell_2 \geq \ell > \ell_1$, $\sigma_{\ell_2}$ fixes every element of $Y(\ell,\alpha)$, while $Y(\ell,\alpha)$ is invariant under the action of $\sigma_{\ell_1}$.  Hence we have $(\sigma_{\ell_1} \circ \sigma_{\ell_2})(x) = \sigma_{\ell_1}(x) = (\sigma_{\ell_2} \circ \sigma_{\ell_1})(x)$.
        \item When $\ell_1 \geq \ell$, we also have $\ell_2 \geq \ell$, therefore both $\sigma_{\ell_1}$ and $\sigma_{\ell_2}$ fix $x$.  Hence we have $(\sigma_{\ell_1} \circ \sigma_{\ell_2})(x) = x = (\sigma_{\ell_2} \circ \sigma_{\ell_1})(x)$.
    \end{itemize}
    Hence we have $(\sigma_{\ell_1} \circ \sigma_{\ell_2})(x) = (\sigma_{\ell_2} \circ \sigma_{\ell_1})(x)$ in any case, therefore $\sigma_{\ell_1} \circ \sigma_{\ell_2} = \sigma_{\ell_2} \circ \sigma_{\ell_1}$.
    By this and the argument in the previous paragraph, the group $G$ generated by $\sigma_0,\dots,\sigma_{d-2}$ is commutative and the map $(\mathbb{Z}/p\mathbb{Z})^{d-1} \to G$, $(e_0,e_1,\dots,e_{d-2}) \mapsto \sigma_0^{e_0} \sigma_1^{e_1} \cdots \sigma_{d-2}^{e_{d-2}}$ is a surjective group homomorphism.
    Hence by the isomorphism theorem for groups, the order $|G|$ of $G$ is a divisor of $|(\mathbb{Z}/p\mathbb{Z})^{d-1}| = p^{d-1}$, which should be a power of the prime $p$.

    We define the action of $G$ on $X$ by $\tau \cdot \{x_1,\dots,x_m\} := \{\tau(x_1),\dots,\tau(x_m)\}$.
    For the orbit decomposition of $X$ by the action, each orbit has order equal to that of some quotient group of $G$, which is a power of the prime $p$ as well as $|G|$.
    Hence, by considering the set $X_0 := \{ S \in X \mid \mbox{$\tau \cdot S = S$ for any $\tau \in G$} \}$ of the fixed points by the action, any orbit in $X$ involving an element of $X \setminus X_0$ has order divisible by $p$.
    Therefore we have $|X| \equiv_p |X_0|$.
    The remaining task is to show that $|X_0|$ is equal to the right-hand side of the statement.

    Let $S \in X_0$.
    For $\ell \in [0,d-1]$ and $\alpha \in [0,n_{\ell}-1]$, suppose that $S \cap Y(\ell,\alpha) \neq \emptyset$ and take its element $x$.
    By the argument above, each of $\sigma_0,\dots,\sigma_{\ell-1}$ fixes no element of $Y(\ell,\alpha)$.
    Therefore, by the definitions of these maps, all elements of $Y(\ell,\alpha)$ can be obtained by applying elements of $G$ to $x$, and all of those elements belong to $S$, as $S \in X_0$.
    Therefore, either $S \cap Y(\ell,\alpha) = \emptyset$ or $Y(\ell,\alpha) \subseteq S$ holds.
    This implies that, by putting $I_{\ell} := \{ \alpha \in [0,n_{\ell}-1] \mid Y(\ell,\alpha) \subseteq S \}$, we have $S = \bigcup_{\ell=0}^{d-1} \bigcup_{\alpha \in I_{\ell}} Y(\ell,\alpha)$.
    Conversely, by the argument above, each set $Y(\ell,\alpha)$ is invariant under the action of $G$, therefore any element $S \in X$ of this form belongs to $X_0$.
    This implies that an element of $X_0$ is determined solely by the choices of sets $I_{\ell}$.
    Now put $c_{\ell} := |I_{\ell}|$.
    Then, as $|Y(\ell,\alpha)| = p^{\ell}$, the corresponding element $S \in X_0$ satisfies that $|S| = \sum_{\ell=0}^{d-1} c_{\ell} p^{\ell} = (c_{d-1} \cdots c_1 c_0)_p$.
    The latter value is equal to $|S| = m$ if and only if $c_{\ell} = m_{\ell}$ holds for every $\ell$.
    This implies that $|X_0|$ is equal to the number of choices of $m_{\ell}$ elements for $I_{\ell}$ from the $n_{\ell}$-element set $[0,n_{\ell}-1]$ for all $\ell$.
    The latter number is equal to the right-hand side $\binom{ n_{d-1} }{ m_{d-1} } \cdots \binom{ n_1 }{ m_1 } \binom{ n_0 }{ m_0 }$ of the claim, concluding the proof.
\end{proof}


\begin{thebibliography}{9}

    \bibitem{Lucas78}
    E.~Lucas,
    \lq\lq Th\'{e}orie des Fonctions Num\'{e}riques Simplement P\'{e}riodiques\rq\rq,
    Amer.\ J.\ Math.\ \textbf{1}(3) (1878) 197--240

\end{thebibliography}

\end{document}