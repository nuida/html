\documentclass{article}
\usepackage{amssymb}
\usepackage{amsthm}
\newtheorem{lemma}{Lemma}
\newtheorem{proposition}{Proposition}
\newtheorem{theorem}{Theorem}
\title{A note on Tychonoff's Theorem and Axiom of Choice}
\author{Koji Nuida}
\date{\today}
\begin{document}
\maketitle

\begin{abstract}
The aim of this note is to give some observation on a standard proof to deduce Axiom of Choice from Tychonoff's Theorem.
\end{abstract}

In this note, we basically deal with the axioms $\mathrm{ZF}^-$ of set theory, which means the Zermelo--Fraenkel set theory $\mathrm{ZF}$ except the Axiom of Foundation
\begin{displaymath}
\forall x (\exists y (y \in x) \to \exists y (y \in x \land \exists z (z \in x \land z \in y))) \enspace.
\end{displaymath}

In this note, we say that a class $\mathcal{K}$ of sets is \emph{downward closed} if, for any set $A \in \mathcal{K}$ and any set $B$ for which there exists an injective map $B \hookrightarrow A$, it follows that $B \in \mathcal{K}$.
Intuitively, this means that $\mathcal{K}$ is a class of cardinal numbers with the property that $|X| \leq |Y| \in \mathcal{K}$ implies $|X| \in \mathcal{K}$.
For example, the classes $\mathsf{Set}$ of all sets, $\mathsf{Finite}$ of all finite sets, and $\mathsf{Countable}$ of all countable sets (i.e., sets $A$ for which $|A| \leq \aleph_0$) are downward closed classes of sets.

On the other hand, we say that a class $\mathcal{T}$ of topological spaces is a \emph{topological property} if, for any $X \in \mathcal{T}$ and any topological space $Y$ which is homeomorphic to $X$, it follows that $Y \in \mathcal{T}$.
Namely, we identify a topological property (in usual sense) with the class of all topological spaces having the property.
For example, the classes $\mathsf{Top}$ of all topological spaces, $\mathsf{T}_1$ of all $T_1$-spaces, and $\mathsf{Hausdorff}$ of all Hausdorff spaces are topological properties.
A member of a topological property $\mathcal{T}$ is said to be a \emph{$\mathcal{T}$-space}.

In what follows, we assume that $\mathcal{K}$ is a downward closed class of sets and $\mathcal{T}$ is a topological property.
We define the following propositions:
\begin{description}
\item[$\mathrm{AC}(\mathcal{K})$]
Let $\mathcal{A}$ be a family of non-empty sets with $\mathcal{A} \in \mathcal{K}$.
Then there exists a choice function for $\mathcal{A}$, i.e., a map $f \colon \mathcal{A} \to \bigcup \mathcal{A}$ satisfying that $f(A) \in A$ for every $A \in \mathcal{A}$.
\item[$\mathrm{ACEq}(\mathcal{K})$]
The same as $\mathrm{AC}(\mathcal{K})$, except that all members of $\mathcal{A}$ are supposed to have equal cardinality.
\item[$\mathrm{AMC}$]
Let $\mathcal{A}$ be a family of non-empty sets.
Then there exists a \lq\lq multiple choice function'' for $\mathcal{A}$, i.e., a map $f \colon \mathcal{A} \to 2^{\bigcup \mathcal{A}}$ satisfying that for each $A \in \mathcal{A}$, $f(A)$ is a finite non-empty subset of $A$.
\item[$\mathrm{AMCEq}$]
The same as $\mathrm{AMC}$, except that all members of $\mathcal{A}$ are supposed to have equal cardinality.
\item[$\mathrm{T}(\mathcal{T},\mathcal{K})$]
Let $\mathcal{A}$ be a family of compact $\mathcal{T}$-spaces.
Then any open cover $\mathcal{W}$ of the product topological space $\prod \mathcal{A}$ has a subcover $\mathcal{W}'$ with $\mathcal{W}' \in \mathcal{K}$.
\item[$\mathrm{THomeo}(\mathcal{T},\mathcal{K})$]
The same as $\mathrm{T}(\mathcal{T},\mathcal{K})$, except that all members of $\mathcal{A}$ are supposed to be homeomorphic to each other.
\end{description}
For example, $\mathrm{AC}(\mathsf{Set})$ is the Axiom of Choice (AC), $\mathrm{AC}(\mathsf{Countable})$ is the Axiom of Countable Choice (ACC), $\mathrm{AMC}$ is the Axiom of Multiple Choice (AMC), and $\mathrm{T}(\mathsf{Top},\mathsf{Finite})$ is the Tychonoff's Theorem.
Note that $\mathrm{AC}(\mathsf{Finite})$ is a theorem of $\mathrm{ZF}^-$ (so is $\mathrm{ACEq}(\mathsf{Finite})$); roughly speaking, a finite number of selections can be unconditionally done simultaneously.
Note also that all the above propositions are consequences of AC in $\mathrm{ZF}^-$, since Tychonoff's Theorem can be proven in $\mathrm{ZF}^- + \mathrm{AC}$ (see Appendix below).

Now we describe a \lq\lq pattern'' of a proof (in $\mathrm{ZF}^-$) to deduce AC from \lq\lq Tychonoff-like'' axioms, which is a slight modification of the standard proof to deduce AC from the original Tychonoff's Theorem:
\begin{quotation}
Let $\mathcal{A} = (A_i)_{i \in \Lambda}$ be a family of non-empty sets.
First, choose a set $p$ which does not belong to any $A_i$ (by Russell's Paradox, the union $\bigcup_{i \in \Lambda} A_i$ does not contain all sets).
Now we assume the following:
\begin{description}
\item[(*)] There exists a map which associates to each $i \in \Lambda$ a topological structure on $X_i := A_i \cup \{p\}$ with the property that (I) each $X_i$ becomes a compact $\mathcal{T}$-space, and (II) there exists a map which associates to each $i \in \Lambda$ an open neighborhood $U_i$ of $p$ in $X_i$ with $U_i \neq X_i$.
\end{description}
We introduce a topological structure on each $X_i$ as above.
Now for each $i \in \Lambda$, let $\widetilde{U}_i$ denote the direct product of $U_i$ and all $X_j$ for $j \in \Lambda \smallsetminus \{i\}$.
Then $\mathcal{W} := (\widetilde{U}_i)_{i \in \Lambda}$ is a family of open subsets of $X := \prod_{i \in \Lambda} X_i$.

Assuming the proposition $\mathrm{AC}(\mathcal{K})$, it follows that $\mathcal{W}$ does not have a subfamily $\mathcal{W}' = (\widetilde{U}_i)_{i \in \Lambda'}$ with the property that $\Lambda' \in \mathcal{K}$ and $\mathcal{W}'$ is an open cover of $X$.
Indeed, for such a subfamily $\mathcal{W}'$, $\mathrm{AC}(\mathcal{K})$ implies that there exists an element $g \in \prod_{i \in \Lambda'} (X_i \smallsetminus U_i)$, and now the element $f \in X$ defined by $f(i) = g(i)$ for $i \in \Lambda'$ and $f(i) = p$ for $i \in \Lambda \smallsetminus \Lambda'$ does not belong to any member of $\mathcal{W}'$, a contradiction.

By the above argument, assuming the proposition $\mathrm{T}(\mathcal{T},\mathcal{K})$ further, it follows that $\mathcal{W}$ is not an open cover of $X$.
Namely, there exists an element $f \in X$ that does not belong to any $\widetilde{U}_i$ with $i \in \Lambda$.
This means that we have $f(i) \not\in U_i$, hence $f(i) \neq p$, for each $i \in \Lambda$; therefore $f$ is an element of $\prod_{i \in \Lambda} A_i$.
Hence AC holds.
\end{quotation}
By this argument, the combination of $\mathrm{AC}(\mathcal{K})$, $\mathrm{T}(\mathcal{T},\mathcal{K})$ and a certain condition ensuring the property (*) (if necessary) implies AC in $\mathrm{ZF}^-$.
We consider some special cases:
\begin{itemize}
\item When $\mathcal{T} = \mathsf{Top}$, to ensure (*) it suffices to define the open sets of each $X_i$ as $\emptyset$, $X_i$ and $\{p\}$.
Indeed, the condition (I) is satisfied, while the condition (II) is also satisfied by defining $U_i = \{p\}$.
As a result, $\mathrm{AC}(\mathcal{K})$ and $\mathrm{T}(\mathsf{Top},\mathcal{K})$ imply AC.
In particular, $\mathrm{T}(\mathsf{Top},\mathsf{Finite})$ (i.e., Tychonoff's Theorem) implies AC; and $\mathrm{AC}(\mathsf{Countable})$ (i.e., ACC) and $\mathrm{T}(\mathsf{Top},\mathsf{Countable})$ (\lq\lq the product of compact spaces is a Lindel\"{o}f space'') also imply AC.
Note also that this argument proves stronger results such as that Tychonoff's Theorem for topological spaces \emph{with precisely three open sets} implies AC.
\item When $\mathcal{T} = \mathsf{T}_1$, to ensure (*) it suffices to first introduce the cofinite topology on each $A_i$ and then attach to $A_i$ a point $p$ as an isolated point.
Indeed, the cofinite topology is a compact $T_1$ topology, therefore the condition (*) is satisfied by choosing $U_i = \{p\}$ for (II) again.
As a result, $\mathrm{AC}(\mathcal{K})$ and $\mathrm{T}(\mathsf{T}_1,\mathcal{K})$ imply AC.
In particular, $\mathrm{T}(\mathsf{T}_1,\mathsf{Finite})$ (i.e., Tychonoff's Theorem for $T_1$-spaces) implies AC; and $\mathrm{AC}(\mathsf{Countable})$ (i.e., ACC) and $\mathrm{T}(\mathsf{T}_1,\mathsf{Countable})$ (\lq\lq the product of compact $T_1$-spaces is a Lindel\"{o}f space'') also imply AC.
We emphasize that this definition of topology on $X_i$ is adopted for a proof of AC from Tychonoff's Theorem in several books, but in fact the argument shows a stronger property that Tychonoff's Theorem \emph{for $T_1$-spaces} implies AC (and, as mentioned above, a simpler choice of trivial (or indiscrete) topology on $A_i$ instead of cofinite topology is enough to prove that Tychonoff's Theorem implies AC).
\item On the other hand, when $\mathcal{T} = \mathsf{Hausdorff}$, a similar strategy to first introduce a compact Hausdorff topology on each $A_i$ and then attach an isolated point $p$ \emph{cannot succeed}.
Indeed, if it is possible, then Tychonoff's Theorem for Hausdorff spaces (i.e., $\mathrm{T}(\mathsf{Hausdorff},\mathsf{Finite})$) could imply AC, but this has been proven as impossible.
For an alternative strategy, here we introduce the discrete topology on each $A_i$, and then define $X_i$ to be the one-point compactification of $A_i$.
In this case, the open neighborhoods of $p$ in $X_i$ are complements in $X_i$ of finite subsets of $A_i$.
The problem is that there is yet no clue to choose a distinguished open neighborhood of $p$ in each $X_i$ (except $X_i$ itself).
Now we introduce an additional axiom $\mathrm{AMC}$, which enables us to choose a distinguished finite non-empty subset $B_i$ of each $A_i$, hence an open neighborhood $U_i = X_i \smallsetminus B_i$ of $p$ in each $X_i$, as desired.
(Note that $\mathrm{AMC}$ is known to be strictly weaker than $\mathrm{AC}$ in $\mathrm{ZF}^-$.)
As a result, the combination of $\mathrm{AMC}$, $\mathrm{AC}(\mathcal{K})$ and $\mathrm{T}(\mathsf{Hausdorff},\mathcal{K})$ implies AC in $\mathrm{ZF}^-$.
In particular, $\mathrm{AMC}$ and Tychonoff's Theorem for Hausdorff spaces imply AC; and $\mathrm{AMC}$, $\mathrm{ACC}$ and \lq\lq the product of compact Hausdorff spaces is a Lindel\"{o}f space'' also imply AC.
\end{itemize}

Moreover, we consider further relaxation of the assumptions in the above argument.
The key fact is the following:
\begin{lemma}
\label{lem:product_of_equal_cardinality}
Let $\mathcal{A}$ be a family of non-empty sets.
Then there exists a non-empty set $B$ satisfying that the sets $A \times B$ for $A \in \mathcal{A}$ have equal cardinality.
\end{lemma}
\begin{proof}
We define $B := (\bigcup \mathcal{A})^{< \omega}$ ($\omega$ denoting the least infinite ordinal number).
Let $A \in \mathcal{A}$.
Then for each element $\xi = (a,(x_0,x_1,\dots,x_n))$ of $A \times B$, define $f(\xi) = (x_0,x_1,\dots,x_n,a) \in B$.
We show that $f \colon A \times B \to B$ is injective.
If $f((a,(x_0,\dots,x_n))) = f((a',(x'_0,\dots,x'_m)))$, then we have $(x_0,\dots,x_n,a) = (x'_0,\dots,x'_m,a')$, therefore $n = m$, $a = a'$ and $x_i = x'_i$ for every $0 \leq i \leq n$.
Hence $f$ is injective, therefore $|A \times B| \leq |B|$, while obviously $|B| \leq |A \times B|$ (since $A$ is non-empty).
Now Cantor--Bernstein--Schroeder Theorem implies that $|A \times B| = |B|$ for every $A \in \mathcal{A}$.
\end{proof}
By using Lemma \ref{lem:product_of_equal_cardinality}, we modify the above pattern of a proof to deduce AC in the following manner:
\begin{enumerate}
\item For a family $\mathcal{A} = (A_i)_{i \in \Lambda}$ of non-empty sets, first choose a non-empty set $B$ with the property that the sets $A'_i := A_i \times B \neq \emptyset$ for $i \in \Lambda$ have equal cardinality (by using Lemma \ref{lem:product_of_equal_cardinality}).
\item Secondly, construct compact $\mathcal{T}$-spaces $X_i = A'_i \cup \{p\}$ and open neighborhoods $U_i \subsetneq X_i$ of $p$ in the same way as (*), with an additional requirement that the $X_i$ for $i \in \Lambda$ are homeomorphic to each other.
\item By assuming $\mathrm{AC}(\mathcal{K})$ or some weakened variant, prove that $\mathcal{W} = (\widetilde{U}_i)_{i \in \Lambda}$ does not have a subfamily $\mathcal{W}' = (\widetilde{U}_i)_{i \in \Lambda'}$ with the property that $\Lambda' \in \mathcal{K}$ and $\mathcal{W}'$ is an open cover of $X$.
\item Finally, by assuming $\mathrm{THomeo}(\mathcal{T},\mathcal{K})$, deduce that $\mathcal{W}$ is not an open cover of $X$, yielding an element $f$ of $\prod_{i \in \Lambda} A'_i$.
Then we obtain an element of $\prod_{i \in \Lambda} A_i$ by taking the first component of each $f(i)$, $i \in \Lambda$.
\end{enumerate}
In the special cases that $\mathcal{T} = \mathsf{Top}$ and $\mathcal{T} = \mathsf{T}_1$ discussed above, the definitions of topology on each $X_i$ satisfy that, for each $i,j \in \Lambda$, the extension of a bijection $A'_i \to A'_j$ to a map $X_i \to X_j$ defined by $p \mapsto p$ gives a homeomorphism $X_i \to X_j$.
Moreover, since now $U_i = \{p\}$, the components of the direct product $\prod_{i \in \Lambda'} (X_i \smallsetminus U_i) = \prod_{i \in \Lambda'} A'_i$ have equal cardinality.
Hence, for the choice of $\mathcal{T}$, the combination of weakened propositions $\mathrm{ACEq}(\mathcal{K})$ and $\mathrm{THomeo}(\mathcal{T},\mathcal{K})$ also implies AC.
In particular, Tychonoff's Theorem for \emph{homeomorphic} $T_1$ spaces implies AC.

In contrast, for the other special case that $\mathcal{T} = \mathsf{Hausdorff}$, the modified argument proves that $\mathrm{AMCEq}$, $\mathrm{AC}(\mathcal{K})$ and $\mathrm{THomeo}(\mathsf{Hausdorff},\mathcal{K})$ imply AC, but \emph{not} that $\mathrm{AMCEq}$, $\mathrm{ACEq}(\mathcal{K})$ and $\mathrm{THomeo}(\mathsf{Hausdorff},\mathcal{K})$ imply AC.
This is because the finite subsets $B_i$ of the sets $A'_i$ obtained by applying $\mathrm{AMCEq}$ are not necessarily of the same size, therefore the direct product $\prod_{i \in \Lambda'} (X_i \smallsetminus U_i)$ in the argument does not necessarily satisfy the hypothesis of $\mathrm{ACEq}(\mathcal{K})$.
If $\mathrm{AMCEq}$ is strengthened in such a way that each component of the direct product will have a finite non-empty distinguished subset \emph{of equal size}, then the strengthened variant of $\mathrm{AMCEq}$ and two axioms $\mathrm{ACEq}(\mathcal{K})$ and $\mathrm{THomeo}(\mathsf{Hausdorff},\mathcal{K})$ imply AC.
I do not know whether or not the combination of $\mathrm{AMCEq}$, $\mathrm{ACEq}(\mathcal{K})$ and $\mathrm{THomeo}(\mathsf{Hausdorff},\mathcal{K})$ can imply AC in $\mathrm{ZF}^-$.

\section*{Appendix: Proof of Tychonoff's Theorem from Axiom of Choice}

In this appendix, we show one of the standard proofs of Tychonoff's Theorem from Axiom of Choice, for the sake of clarifying that the proof indeed works in $\mathrm{ZF}^-$.
The proof is taken from Section 16 of \cite{AC_math}.

First, we notice the following equivalent form of Axiom of Choice, called Tukey's Lemma.
We prepare a terminology.
We say that a family $\mathcal{F}$ of sets is of \emph{finite character} if, for any set $A$, we have $A \in \mathcal{F}$ if and only if every finite subset of $A$ belongs to $\mathcal{F}$.
Then the following fact is known:
\begin{theorem}
[see e.g., {\cite[Exercise 11 in Chapter I]{Kunen}}]
\label{thm:Tukey}
In $\mathrm{ZF}^-$, $\mathrm{AC}$ is equivalent to the following proposition (\emph{Tukey's Lemma}): For any family $\mathcal{F}$ of finite character and any $A \in \mathcal{F}$, there exists a maximal member (with respect to inclusion) of $\mathcal{F}$ containing $A$.
\end{theorem}

We start the proof of Tychonoff's Theorem.
Let $X = \prod_{i \in \Lambda} X_i$ be the product space of compact topological spaces $X_i$.
It suffices to show that, for any family $\mathcal{F}$ of subsets of $X$ having finite intersection property, the intersection of the family $\overline{\mathcal{F}} := \{\overline{A} \mid A \in \mathcal{F}\}$ is non-empty (where $\overline{A}$ denotes the closure of $A$).
First, note that the collection of the families $\mathcal{F}$ satisfying the above condition is of finite character, therefore by using Tukey's Lemma, there exists a maximal family subject to this condition that contains a given family.
Hence we may assume without loss of generality that a given family $\mathcal{F}$ itself is maximal.

For each $i \in \Lambda$, let $\mathcal{F}_i$ be the collection of the closures $\overline{\pi_i[A]}$ in $X_i$ of the images $\pi_i[A]$ of all $A \in \mathcal{F}$ by the projection $\pi_i \colon X \to X_i$.
Since $\mathcal{F}$ has finite intersection property, $\mathcal{F}_i$ also has finite intersection property (note that $\pi_i[\bigcap_{k=1}^{n} A_k] \subset \bigcap_{k=1}^{n} \pi_i[A_k] \subset \bigcap_{k=1}^{n} \overline{\pi_i[A_k]}$ for a finite number of $A_k \in \mathcal{F}$).
Since $X_i$ is compact, we have $\bigcap \mathcal{F}_i \neq \emptyset$; choose (by using $\mathrm{AC}$) an element $p_i \in \bigcap \mathcal{F}_i$ for each $i \in \Lambda$.
We show that the element $p = (p_i)_{i \in \Lambda}$ of $X$ belongs to $\bigcap \overline{\mathcal{F}}$.
This is equivalent to that each open neighborhood $U$ of $p$ in $X$ intersects with every $A \in \mathcal{F}$.
Moreover, it suffices to prove the claim for the case that $U$ belongs to the open basis of $X$, namely there exist a finite subset $\Lambda'$ of $\Lambda$ and an open neighborhood $U_i$ of $p_i$ in $X_i$ for each $i \in \Lambda'$ with the property that $U$ is the direct product of the $U_i$ for $i \in \Lambda'$ and $X_i$ for $i \in \Lambda \smallsetminus \Lambda'$.

For each $i \in \Lambda'$, let $W_i$ denote the direct product of $U_i$ and all $X_j$ for $j \in \Lambda \smallsetminus \{i\}$.
Then we have $U = \bigcap_{i \in \Lambda'} W_i$.
Now for each $A \in \mathcal{F}$, we have $p_i \in U_i \cap \overline{\pi_i[A]}$ by the choice of $p$, therefore $U_i \cap \pi_i[A] \neq \emptyset$.
By the definition of $W_i$, this implies that $W_i \cap A \neq \emptyset$ for every $A \in \mathcal{F}$.
Now, since $\mathcal{F}$ is maximal, we have the following properties:
\begin{itemize}
\item For a finite number of members $A_k$ of $\mathcal{F}$, we have $\bigcap_k A_k \in \mathcal{F}$; this is because $\mathcal{F} \cup \{\bigcap_k A_k\}$ has finite intersection property as well as $\mathcal{F}$.
\item For each $i \in \Lambda'$, $\mathcal{F} \cup \{W_i\}$ has finite intersection property; this is because, for a finite number of members $A_k$ of $\mathcal{F}$, we have $\bigcap_k A_k \in \mathcal{F}$ by the above argument, therefore $W_i \cap \bigcap_k A_k \neq \emptyset$ as shown above.
\end{itemize}
Hence, since $\mathcal{F}$ is maximal, it follows that $W_i \in \mathcal{F}$ for every $i \in \Lambda'$.
Now for each $A \in \mathcal{F}$, we have $\emptyset \neq A \cap \bigcap_{i \in \Lambda'} W_i = A \cap U$ by the finite intersection property of $\mathcal{F}$.
Hence $U$ intersects with every $A \in \mathcal{F}$, as desired.
This completes the proof.

\begin{thebibliography}{9}

\bibitem{AC_math}
H.~Tanaka,
\lq\lq Axiom of Choice and Mathematics,'' Enlarged Edition (in Japanese),
Yuseisha, 1999.

\bibitem{Kunen}
K.~Kunen,
\lq\lq SET THEORY, An introduction to independence proofs,''
Elsevier, 1980.

\end{thebibliography}

\end{document}