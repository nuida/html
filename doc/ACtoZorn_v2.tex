\documentclass{article}
\usepackage{amssymb}
\usepackage{amsthm}
\newtheorem{lemma}{Lemma}
\newtheorem{theorem}{Theorem}
\title{A note on a proof of Zorn's Lemma from Axiom of Choice without transfinite induction}
\author{Koji Nuida}
\date{\today}
\begin{document}
\maketitle

\begin{abstract}
In this note, we give a proof of Zorn's Lemma from Axiom of Choice without transfinite induction (whose essential idea is the same as the proof of Theorem 4.19 in the book [H.~Rubin and J.~E.~Rubin, \lq\lq Equivalents of the Axiom of Choice, II'', Second Edition, Studies in Logic and the Foundations of Mathematics vol.116, North-Holland, 1985], as I noticed after writing the note ...).
For the sake of comparison, we also include as an appendix a proof of the claim using transfinite induction.
\end{abstract}

Throughout this note, let $(X,\leq)$ denote an arbitrary non-empty partially ordered set in which every chain (linearly or totally ordered subset) $C$ has an upper bound (with respect to $\leq$), that is, an element $x \in X$ satisfying that $c \leq x$ for every $c \in C$.
Then Zorn's Lemma states that such an $X$ always has a maximal element (with respect to $\leq$), that is, an element $x \in X$ satisfying that there exist no elements $y \in X$ with $x < y$.
In this note, we give a proof of Zorn's Lemma from Axiom of Choice (in Zermelo--Fraenkel set theory), without transfinite induction which would be used in a \lq\lq natural'' proof of the claim.

Assume, for the contrary, that an $X$ satisfying the hypothesis has no maximal elements.
We will derive a contradiction from this assumption.

We summarize some basic definitions and terminologies:
\begin{itemize}
\item We say that a chain $C$ in a subset $Y$ of $X$ is \emph{bounded in $Y$} if there exists an element $x \in Y \setminus C$ satisfying that $c < x$ for every $c \in C$, and \emph{unbounded in $Y$} otherwise.
\item A chain $C$ in $X$ is called \emph{well-ordered} if any non-empty subset of $C$ has a unique minimal element.
\item For a chain $C$ and an element $c \in C$, the subset $\{d \in C \mid d < c\}$ is called the \emph{initial segment} of $C$ relative to $c$ and is denoted by $s_C(c)$.
\end{itemize}

Let $\mathcal{C}$ denote the set of all non-empty chains in $X$.
For each $C \in \mathcal{C}$, define $U_C = \{x \in X \setminus C \mid c < x \mbox{ for every } c \in C\}$.
We notice the following property:
\begin{lemma}
\label{lem:chain_has_proper_upper_bound}
We have $U_C \neq \emptyset$ for every $C \in \mathcal{C}$.
\end{lemma}
\begin{proof}
By the hypothesis of Zorn's Lemma, $C$ has an upper bound $x \in X$.
If $y \in U_{\{x\}}$, then we have $c \leq x < y$ (hence $c < y$) for every $c \in C$, therefore $y \not\in C$ and $y \in U_C$.
Hence we have $U_{\{x\}} \subset U_C$, while $U_{\{x\}} \neq \emptyset$ by the assumption that $X$ has no maximal elements, therefore we have $U_C \neq \emptyset$.
\end{proof}
We would like to construct (by using Axiom of Choice) a choice function $f$ of the family $\{U_C\}_{C \in \mathcal{C}}$ which satisfies some more appropriate properties described below.

For an arbitrary pair of $C_1,C_2 \in \mathcal{C}$, we define a symmetric relation $C_1 \sim_{\mathrm{pre}} C_2$ to mean that the chain $C_1 \cap C_2$ is unbounded in both $C_1$ and $C_2$.
Then we define an equivalence relation $\sim$ on $\mathcal{C}$ to be the transitive closure of $\sim_{\mathrm{pre}}$, that is, we have $C_1 \sim C_2$ if and only if there exists a finite sequence $C_1 = C'_0,C'_1,\dots,C'_n = C_2$ ($n \geq 0$) of elements of $\mathcal{C}$ with $C'_i \sim_{\mathrm{pre}} C'_{i+1}$ for every $0 \leq i \leq n-1$.
Note that if $C_1 \sim C_2$ and $C_1$ has a maximum element $c$, then $c \in C_2$ and $c$ is also a maximum element of $C_2$.
Now we present the following lemma:
\begin{lemma}
\label{lem:common_upper_bound_for_equivalent_chains}
If $C_1,C_2 \in \mathcal{C}$ and $C_1 \sim C_2$, then $U_{C_1} = U_{C_2}$.
\end{lemma}
\begin{proof}
It suffices to show that $U_{C_1} \subset U_{C_2}$ if $C_1 \sim_{\mathrm{pre}} C_2$.
Let $x \in U_{C_1}$.
If $x \in C_2$, then the upper bound $x$ ($\in C_2 \setminus (C_1 \cap C_2)$) of $C_1$ is also an upper bound of $C_1 \cap C_2$, therefore $C_1 \cap C_2$ is bounded in $C_2$, contradicting the hypothesis $C_1 \sim_{\mathrm{pre}} C_2$.
Hence $x \not\in C_2$.
If $c \in C_2$ and $c \nless x$, then, as $C_1 \cap C_2$ is unbounded in $C_2$, there exists a $d \in C_1 \cap C_2$ with $c \leq d \nless x$.
Now we have $d \in C_1$ and $d \nless x$, contradicting the property $x \in U_{C_1}$.
Hence we have $c < x$ for every $c \in C_2$, therefore $x \in U_{C_2}$.
\end{proof}
We define a subset $U_{\mathcal{E}}$ of $X$ for each $\sim$-equivalence class $\mathcal{E}$ by $U_{\mathcal{E}} = U_C$, where $C \in \mathcal{E}$.
This is well-defined by virtue of Lemma \ref{lem:common_upper_bound_for_equivalent_chains}.
Now we obtain (\underline{by using Axiom of Choice}) a choice function $f$ of the family $\{U_{\mathcal{E}}\}_{\mathcal{E} \in X/{\sim}}$.
Moreover, for simplicity, we write $f(C) = f([C])$ for any $C \in \mathcal{C}$, where $[C]$ denotes the $\sim$-equivalence class of $C$.
Hence $f(C) \in U_C$ for any $C \in \mathcal{C}$.
Moreover, if $C_1,C_2 \in \mathcal{C}$ and $C_1 \sim C_2$, then we have $f(C_1) = f(C_2)$.

From now, we fix an element $x_0 \in X$.
We give the following definition:
\begin{itemize}
\item We say that a $C \in \mathcal{C}$ is \emph{$f$-consecutive} if $C$ is well-ordered, $x_0$ is the minimum element of $C$, and we have $f(s_C(c)) = c$ for any $c \in C \setminus \{x_0\}$.
\end{itemize}
We present some properties of $f$-consecutive chains:
\begin{lemma}
\label{lem:saturated_then_closed}
Let $C \in \mathcal{C}$ be $f$-consecutive, and suppose that $\emptyset \neq C' \subset C$ and $C'$ is bounded in $C$.
Then we have $f(C') = d$, where $d$ is the unique minimal element of the set of all upper bounds $x \in C \setminus C'$ of $C'$.
\end{lemma}
\begin{proof}
Note that $C' \in \mathcal{C}$.
Let $d$ be as specified in the statement, which indeed exists by virtue of the fact that $C$ is well-ordered and $C'$ is bounded in $C$.
We show that $s_C(d) \sim C'$.
First, we have $C' \subset s_C(d)$, therefore $s_C(d) \cap C' = C'$ is unbounded in $C'$.
On the other hand, if $x \in s_C(d) \setminus C'$, then $x$ is not an upper bound of $C'$ by the choice of $d$.
Therefore $C'$ is unbounded in $s_C(d)$.
Hence we have $s_C(d) \sim C'$.
Now we have $f(C') = f(s_C(d))$ by the property of $f$, while we have $f(s_C(d)) = d$, as $C$ is $f$-consecutive.
This implies that $f(C') = d$.
\end{proof}
\begin{lemma}
\label{lem:trichotomy}
Let $C_1,C_2 \in \mathcal{C}$ be two $f$-consecutive chains.
Then exactly one of the following three conditions is satisfied:
\begin{enumerate}
\item $C_1 = C_2$;
\item $C_1$ is an initial segment of $C_2$;
\item $C_2$ is an initial segment of $C_1$.
\end{enumerate}
\end{lemma}
\begin{proof}
It is obvious that two of the three conditions specified in the statement do not occur simultaneously.

First we show that for each $i \in \{1,2\}$, the conditions $x \in C_1 \cap C_2$ and $y \in s_{C_i}(x)$ imply $y \in C_1 \cap C_2$.
Assume, for the contrary, that a counterexample $y$ exists for some $x$, and let $y$ be the minimal counterexample relative to the fixed $x$ (which exists, as $C_i$ is well-ordered).
Then we have $s_{C_i}(y) \neq \emptyset$, as $x_0 \in C_1 \cap C_2$.
By the choice of $y$, we have $s_{C_i}(y) \subset C_{3-i}$ and $y \not\in C_{3-i}$.
We have $f(s_{C_i}(y)) = y$, as $C_i$ is $f$-consecutive.
On the other hand, as $x \in C_1 \cap C_2$ and $y < x$, the subset $s_{C_i}(y)$ of $C_{3-i}$ is bounded in $C_{3-i}$, therefore we have $f(s_{C_i}(y)) \in C_{3-i}$ by Lemma \ref{lem:saturated_then_closed}.
But this contradicts the property $y \not\in C_{3-i}$.
Hence the claim of this paragraph holds.
Moreover, as $C_1$ and $C_2$ are well-ordered, the above fact implies that $C_1 \cap C_2$ is either equal to or an initial segment of $C_i$ for each $i \in \{1,2\}$ (if $C_1 \cap C_2 \neq C_i$, consider the minimum element of $C_i \setminus (C_1 \cap C_2)$).

Now assume, for the contrary, that $C_1 \cap C_2 \neq C_1$ and $C_1 \cap C_2 \neq C_2$.
Then $C_1 \cap C_2$ is an initial segment of both $C_1$ and $C_2$, therefore we have $f(C_1 \cap C_2) \in C_1 \cap C_2$, as $C_1$ and $C_2$ are $f$-consecutive.
But this contradicts the definition of $f$.
Hence we have $C_1 \cap C_2 = C_i$ for some $i \in \{1,2\}$, and now the above result implies that $C_i = C_{3-i}$ or $C_i$ is an initial segment of $C_{3-i}$.
\end{proof}

Let $C_0$ denote the union of all $f$-consecutive chains (note that at least one $f$-consecutive chain, say $\{x_0\}$, exists).
Then Lemma \ref{lem:trichotomy} implies that $C_0 \in \mathcal{C}$ (for two elements $x,y \in C_0$, there exist $f$-consecutive chains $C$ containing $x$ and $C'$ containing $y$, and $x \in C \subset C'$ or $y \in C' \subset C$ by the lemma).
Now we present the following lemma:
\begin{lemma}
\label{lem:relation_of_initial_segments}
Let $c \in C_0$, and let $C$ be an $f$-consecutive chain containing $c$.
Then $s_{C_0}(c) = s_C(c)$.
\end{lemma}
\begin{proof}
First note that $s_C(c) \subset s_{C_0}(c)$, as $C \subset C_0$ by the definition of $C_0$.
We show that $d \in C$ for an arbitrary element $d \in s_{C_0}(c)$.
By the definition of $C_0$, there exists an $f$-consecutive chain $C'$ such that $d \in C'$.
Now we have $d \in C' \subset C$ if $C' = C$ or $C'$ is an initial segment of $C$.
On the other hand, if $C$ is an initial segment of $C'$, then the properties $c \in C$, $d \in C'$ and $d < c$ imply $d \in C$.
Hence we have $d \in C$ by Lemma \ref{lem:trichotomy}.
\end{proof}
From now, we will show that $C_0$ itself is $f$-consecutive.
It is obvious that $x_0$ is the minimum element of $C_0$.
\begin{lemma}
\label{lem:union_is_well-ordered}
$C_0$ is well-ordered.
\end{lemma}
\begin{proof}
Let $A$ be an arbitrary non-empty subset of $C_0$.
Choose an element $a \in A$, and let $C$ be an $f$-consecutive chain with $a \in C$ (which exists by the definition of $C_0$).
As $C$ is well-ordered, $A \cap C$ has the minimum element, say $a_0$.
Now if $a' \in A$ and $a' < a_0$, then we have $a' \in s_{C_0}(a_0) = s_C(a_0)$ by Lemma \ref{lem:relation_of_initial_segments}, therefore $a' \in A \cap C$ and $a' < a_0$, contradicting the choice of $a_0$.
Hence we have $a' \geq a_0$ for every $a' \in A$, therefore $A$ itself has the minimum element $a_0$.
\end{proof}
\begin{lemma}
\label{lem:union_is_closed}
We have $f(s_{C_0}(c)) = c$ for any $c \in C_0 \setminus \{x_0\}$.
\end{lemma}
\begin{proof}
Let $C$ be an $f$-consecutive chain with $c \in C$ (which exists by the definition of $C_0$).
Then we have $s_{C_0}(c) = s_C(c)$ by Lemma \ref{lem:relation_of_initial_segments}.
As $C$ is $f$-consecutive, now we have $f(s_{C_0}(c)) = f(s_C(c)) = c$.
\end{proof}
By Lemma \ref{lem:union_is_well-ordered} and Lemma \ref{lem:union_is_closed}, it follows that $C_0$ is $f$-consecutive.
Now it is straightforward to verify that $C_0 \cup \{f(C_0)\}$ is also an $f$-consecutive chain, but it is not a subset of $C_0$, contradicting the definition of $C_0$.
Hence $X$ should have a maximal element, concluding the proof of Zorn's Lemma.

\section*{Appendix: A proof using transfinite induction}

In this appendix, for the sake of comparison, we describe a proof of Zorn's Lemma from Axiom of Choice using transfinite induction.
First we clarify the statement of the principle for \lq\lq definition by transfinite recursion'' (see e.g., Theorem 9.3 in Chapter I of [K.~Kunen, \lq\lq SET THEORY, An Introduction to Independence Proofs'',  Elsevier, 1980]):
\begin{theorem}
\label{thm:transfinite_induction}
Let $\varphi(x,y)$ be a formula (in Zermelo--Fraenkel set theory) with free variables $x,y$ satisfying $\forall x \exists! y \varphi(x,y)$.
Then there exists a formula $\Phi(x,y)$ with free variables $x,y$ satisfying the following two conditions;
\begin{enumerate}
\item $\forall x ( (x \in \mathbf{ON} \to \exists! y \Phi(x,y)) \land (\neg x \in \mathbf{ON} \to \neg\exists y \varphi(x,y) ) )$;
\item $\forall x ( x \in \mathbf{ON} \to \forall y,z ( y = \Phi\!\upharpoonright_x \land \varphi(y,z) \to \Phi(x,z) ) )$,
\end{enumerate}
where \lq\lq $x \in \mathbf{ON}$'' is an abbreviation of \lq\lq $x$ is an ordinal number'' and \lq\lq $\Phi\!\upharpoonright_x$'' is an abbreviation of the set $\{\langle a,b \rangle \mid a \in x \land \Phi(a,b)\}$ (with $\langle a,b \rangle$ denoting the ordered pair of $a$ and $b$).
\end{theorem}
Intuitively, the theorem means that, if we would like to define a \lq\lq function'' $\Phi$ with domain consisting of all ordinal numbers (the whole of which is never a set) in such a way that the value of $\Phi$ at each ordinal number $\alpha$ is determined by a given rule from the values of $\Phi$ at ordinal numbers less than $\alpha$, then there indeed exists such a \lq\lq function'' $\Phi$.
Note that this is a theorem of ZF set theory and does not depend on Axiom of Choice.

Now we give a proof of Zorn's Lemma from Axiom of Choice using Theorem \ref{thm:transfinite_induction} (as well as transfinite induction).
Let $X \neq 0$ be a partially ordered set appeared in the statement of Zorn's Lemma.
Assume, for the contrary, that $X$ has no maximal elements.
Then, for each non-empty subset $C$ of $X$ which is isomorphic to an ordinal number (hence a chain), it follows \underline{from Axiom of Choice} that there exists a distinguished upper bound $b_C$ of $C$ with $b_C \in X \setminus C$.

To apply Theorem \ref{thm:transfinite_induction}, first we define a formula $\varphi(x,y)$ in the following manner, where we fix an element $a \in X$ throughout the proof:
\begin{itemize}
\item If $x = 0$ ($= \emptyset$), then let $\varphi(x,y)$ mean that $y = a$.
\item If $x$ is a function from an ordinal number $\alpha > 0$ to $X$ which is an isomorphism (between partially ordered sets) onto the image $\mathrm{Im}(x)$ of $x$, then let $\varphi(x,y)$ mean that $y = b_{\mathrm{Im}(x)}$ (note that $\mathrm{Im}(x)$ is isomorphic to the non-empty ordinal number $\alpha$, therefore $b_{\mathrm{Im}(x)}$ is indeed defined).
\item Otherwise, let $\varphi(x,y)$ mean that $y = 0$.
\end{itemize}
This formula $\varphi(x,y)$ satisfies the hypothesis of Theorem \ref{thm:transfinite_induction}, therefore a formula $\Phi(x,y)$ as in the theorem exists.
Now we have the following lemma:
\begin{lemma}
\label{lem:appendix_property_of_Phi}
Let $x$ be an ordinal number, and let $x'$ be the unique element satisfying $\Phi(x,x')$.
\begin{enumerate}
\item We have $x' \in X$.
\item If $y < x$ and $\Phi(y,y')$, then $y' < x'$ in $X$.
\end{enumerate}
\end{lemma}
\begin{proof}
We prove the claim by transfinite induction on $x$.
First, if $x = 0$, then it follows from the definition of the formula $\varphi$ that $x' = a$, therefore the specified conditions are satisfied.
Secondly, suppose that $x > 0$.
Then, by the hypothesis of the transfinite induction, the set $\Phi\!\upharpoonright_x$ in the statement of Theorem \ref{thm:transfinite_induction} is an isomorphism from $x$ to a subset of $X$, say, $C$ (note that $x$ is totally ordered).
Now by the definitions of $\Phi$ and $\varphi$, it follows that $x' = b_C$, therefore the specified conditions are satisfied for $x$ (the second condition follows from the property that $b_C \in X \setminus C$ is an upper bound of $C$).
Hence the claim holds.
\end{proof}
By the second property shown in Lemma \ref{lem:appendix_property_of_Phi}, for each element $v \in X$, there exists at most one ordinal number $x$ satisfying $\Phi(x,v)$.
Let $X'$ denote the subset of $X$ defined in such a way that $v \in X'$ if and only if $v \in X$ and $\Phi(x,v)$ for some (or equivalently, a unique) ordinal number $x$.
By the Axiom Schema of Replacement applied to the set $X'$ and the formula $\Phi'(x,y) := \Phi(y,x)$, there exists a set $Y$ for which we have $y \in Y$ if $y$ is an ordinal number and the unique element $y'$ satisfying $\Phi(y,y')$ belongs to $X'$.
Now by the first property shown in Lemma \ref{lem:appendix_property_of_Phi}, the set $Y$ contains every ordinal number.
However, this contradicts Burali--Forti Paradox (which states that there exist no sets containing all ordinal numbers).
Hence $X$ should have a maximal element, concluding the proof of Zorn's Lemma.
\end{document}